\documentclass{homework}

\title{Homework 1}

\begin{document}

\maketitle

\begin{problem}{1}
Let $g : [a, b] \to [b - a, a - b]$ be a function such that $g(x) = f(x) - a - (x - a)$. $g(a) = f(a) - a \geq 0$, $g(b) = f(b) - a - b + a = f(b) - b \leq 0$. Apply the intermediate value theorem and we obtain that there exists $c \in [a, b]$ such that $g(c) = 0$. For this $c$ we have $g(c) = f(c) - a - c + a = f(c) - c = 0$, or $f(c) = c$. Therefore, there exists $c \in [a, b]$ such that $f(c) = c$. \QED
\end{problem}

\begin{problem}{2}
\begin{enumerate}
\item First we prove that $\sqrt{2}$ is an upper bound of $X$. Assuming that there exists $c \in X$ such that $c > \sqrt{2}$, since $c$ is nonnegative, $c^2 = c \cdot c > \sqrt{2}c > \sqrt{2} \times \sqrt{2} > 2$, which contradicts to the premise that $c^2 < 2$. Therefore, all $c \in X$ is less than $\sqrt{2}$, which proves that $\sqrt{2}$ is an upper bound of $X$. Apply the theorem of the completeness of real numbers and we know that $\sup X$ exists. \QED

\item Since $\sqrt{2}$ is an upper bound of $X$, we know that $\sup X \leq \sqrt{2}$. Since $0 \in X$, $0 \leq \sup X$. Since both $\sup X$ and $\sqrt{2}$ is nonnegative, we can square both and get $a = \sup X \leq 2$. \QED
\end{enumerate}
\end{problem}

\begin{problem}{3}
Let $\{c_n\}_{n \geq 1}$ and $\{d_n\}_{n \geq 1}$ be the sequences of partial sums of $a_n$ and $b_n$ respectively. Namely, $c_i = \sum^{i}_{j=1}a_j$, $d_i = \sum^{i}_{j=1}b_j$. Since both $a_n$ and $b_n$ are sequences of nonnegative real numbers, $c_n$ and $d_n$ are nondecreasing sequences. If $b_n$ converges, then $d_n$ also converges, leading to the fact that $d_n$ is bounded. Because for each positive integer $i$, $0 \leq c_i \leq d_i$, we know that $c_n$ is also bounded, which means it converges. Accordingly, $\sum^{\infty}_{n=1}a_n$ converges. \QED
\end{problem}

\begin{problem}{4}
I will only give the form $c = re^{i\theta}$ or $c = r(\cos\theta + i\sin\theta)$, since it's clear that $r$ is the modulus and $\theta$ the polar angle.

\begin{enumerate}
\item $3+2i = \sqrt{13}e^{i\theta}$ where $\theta = \arctan{\frac{2}{3}}$.

\item
$1-i = \sqrt{2}(\cos\frac{3}{4}\pi + i \sin\frac{3}{4}\pi)$

$(1-i)^4 = 4(\cos3\pi + i \sin3\pi) = 4(\cos\pi + i \sin\pi)$

\item $2 + i = \sqrt{5}e^{i\theta}$ where $\theta = \arctan{\frac{1}{2}}$.

\item
$3 + 4i = 5e^{i\theta}$ where $\theta = \arctan{\frac{4}{3}}$.

$\sqrt[\leftroot{7}]{3+4i} = 5/7e^{i\theta}$ where $\theta = \frac{1}{7}(2k\pi + \arctan{\frac{4}{3}})$, $k = 0,1,2,\hdots$

There are at most $7$ distinct $\theta$, namely $k = 0,1,2,3,4,5,6,7$.
\end{enumerate}
\end{problem}

\begin{problem}{5}
\begin{enumerate}
\item The line $y = 1$ in the $(x, y)$-plane is a \emph{closed} set.

Let $L$ denote the line and $L^C$ its complement $\mathbb{R}^2 - L$. Let $p = (x, y)$ be an arbitrary point in $L^C$ and $\epsilon = |y - 1|$. For any point $q = (x', 1) \in L$, $q \notin B_{\epsilon}(p)$, because $|p - q| = \sqrt{(x - x')^2 + (y - 1)^2} \geq \sqrt{(y - 1)^2} = |y - 1| = \epsilon$. Clearly, $y \neq 1$ and $\epsilon > 0$. For any point $q \in B_{\epsilon}(p)$, because $q \in \mathbb{R}^2$ and $q \notin L$, $q \in L^C$. Therefore, $L^C$ is an open set and consequently $L$ is a closed set. \QED

\item The $(y, z)$-plane in $\mathbb{R}^3$ is a \emph{closed} set.

Let $P$ denote the $(y, z)$-plane $\{(0, y, z) | y, z \in \mathbb{R}\}$ and $P^C$ its complement. Let $p = (x, y, z)$ be an arbitrary point in $L^C$ and $\epsilon = |x|$. Clearly, $x \neq 0$ and $\epsilon > 0$. For any point $q = (0, y', z') \in P$, $q \notin B_{\epsilon}(p)$, because $|p - q| = \sqrt{(x - 0)^2 + (y - y')^2 + (z - z')^2} \geq \sqrt{(x - 0)^2} = |x| = \epsilon$. For any point $q \in B_{\epsilon}(p)$, because $q \in \mathbb{R}^3$ and $q \notin P$, $q \in P^C$. Therefore, $P^C$ is an open set and consequently $P$ is a closed set. \QED

\item The unit circle centered at $(0, 0)$ in $\mathbb{R}^2$ is a \emph{closed} set.

Let $S$ denote the unit circle, $S^C$ its complement, and $o = (0, 0)$. Let $p$ be an arbitrary point in $S^C$ and $\epsilon = |1 - |p - o||$. Clearly $|p - o| \neq 1$---otherwise $p$ is on the unit circle---and $\epsilon > 0$. For any point $q \in S$, $q \notin B_{\epsilon}(p)$, because
\begin{enumerate}
\item $|p - o| > |q - o|$:
\begin{align*}
  |p - q| + |q - o| & \geq |p - o| \\
  |p - q| & \geq |p - o| - |q - o| > 0 \\
  |p - q| & \geq |p - o| - 1 = \epsilon
\end{align*}

\item $|q - o| > |p - o|$:
\begin{align*}
  |q - p| + |p - o| & \geq |q - o| \\
  |q - p| & \geq |q - o| - |p - o| > 0 \\
  |q - p| & \geq 1 - |p - o| = \epsilon
\end{align*}
\end{enumerate}
. For any point $q \in B_{\epsilon}(p)$, because $q \in \mathbb{R}^2$ and $q \notin S$, $q \in S^C$. Therefore, $S^C$ is an open set and consequently $S$ is a closed set. \QED

\item The set $\{(x, 0) | x > 0\}$ in $\mathbb{R}^2$ is \emph{neither open nor closed}.

Let $S$ denote the set and $S^C$ its complement $\mathbb{R}^2 - S$.
\begin{enumerate}
\item $S^C$ is not open. Let $p = (0, 0)$ For any $\epsilon > 0$, define $q = (\epsilon/2, 0)$, $q \in B_{\epsilon}(p)$, $q \in S$, namely $q \notin S^C$. \QED

\item $S$ is not open. Let $p = (x, 0)$ be an arbitrary point in $S$. For any $\epsilon > 0$, define $q = (x, \epsilon/2)$, $q \in B_{\epsilon}(p)$, $q \in S^C$, namely $q \notin S$. \QED
\end{enumerate}

Since both $S$ and its complement are not open, $S$ is neither open nor closed. \QED

\item The set $\{(x, 0) | x \geq 0\}$ in $\mathbb{R}^2$ is \emph{closed}.

Let $S$ denote the set and $S^C$ its complement $\mathbb{R}^2 - S$. Let $p$ be an arbitrary point in $S^C$.
\begin{enumerate}
\item $p = (x, y)$ where $y \neq 0$. Let $\epsilon = |y|$, which is clearly positive. For any point $q$ in $S$, $|q - p| \geq \sqrt{(y - 0)^2} = |y| = \epsilon$, so $q \notin B_{\epsilon}(p)$.

\item $p = (x, 0)$. Let $\epsilon = |x|$, which is also clearly positive. For any point $q$ in $S$, $|q - p| \geq \sqrt{(x - 0)^2} = |x| = \epsilon$, so $q \notin B_{\epsilon}(p)$.
\end{enumerate}

Therefore, for any point $q \in B_{\epsilon}(p)$, because $q \in \mathbb{R}^2$ and $q \notin S$, $q \in S^C$. Consequently, $S^C$ is an open set and $S$ is a closed set. \QED

\item The set $\{(x, x) | x \in \mathbb{R}\}$ in $\mathbb{R}^2$ is \emph{closed}.\par
\textbf{Claim}. If $a + b \geq c$, then $a^2 + b^2 \geq \frac{1}{2}c^2$.\par
\textbf{Proof}. Since $a + b \geq c$, $a^2 + b^2 + 2ab \geq c^2$. Since $a^2 + b^2 - 2ab = (a-b)^2 \geq 0$, $a^2 + b^2 > 2ab$. Therefore, $2(a^2 + b^2) \geq a^2 + b^2 + 2ab \geq c^2$, or $a^2 + b^2 \geq \frac{1}{2} c^2$. \QED

Let $L$ denote the set and $L^C$ its complement $\mathbb{R}^2 - L$. Let $p = (x, y)$ be an arbitrary point in $L^C$ and $\epsilon = \sqrt{2}/2|x - y|$. Clearly $\epsilon > 0$. Let  $q = (c, c)$ be an arbitrary point in $L$. Since $|c - x| + |c - y| \geq |x - y|$ (too trivial, proof omitted here), $(c + x)^2 + (c + y)^2 \geq \frac{1}{2}(x - y)^2$. Then, $|p - q| = \sqrt{(c + x)^2 + (c + y)^2} \geq \sqrt{2}/2|x - y| = \epsilon$, i.e. $q \notin B_{\epsilon}(p)$. Therefore, for any point $q \in B_{\epsilon}(p)$, because $q \in \mathbb{R}^2$ and $q \notin L$, $q \in L^C$. Consequently, $L^C$ is an open set and $L$ is a closed set. \QED

\item The sphere of radius $2$ centered at $(0, 0, 0)$ in $\mathbb{R}^3$ is a \emph{closed} set.

Let $S$ denote the sphere, $S^C$ its complement, and $o = (0, 0, 0)$. Let $p$ be an arbitrary point in $S^C$ and $\epsilon = |2 - |p - o||$. Clearly $|p - o| \neq 2$---otherwise $p$ is on the sphere---and $\epsilon > 0$. For any point $q \in S$, $q \notin B_{\epsilon}(p)$, because
\begin{enumerate}
\item $|p - o| > |q - o|$:
\begin{align*}
  |p - q| + |q - o| & \geq |p - o| \\
  |p - q| & \geq |p - o| - |q - o| > 0 \\
  |p - q| & \geq |p - o| - 2 = \epsilon
\end{align*}

\item $|q - o| > |p - o|$:
\begin{align*}
  |q - p| + |p - o| & \geq |q - o| \\
  |q - p| & \geq |q - o| - |p - o| > 0 \\
  |q - p| & \geq 2 - |p - o| = \epsilon
\end{align*}
\end{enumerate}
. For any point $q \in B_{\epsilon}(p)$, because $q \in \mathbb{R}^3$ and $q \notin S$, $q \in S^C$. Therefore, $S^C$ is an open set and consequently $S$ is a closed set. \QED

\item $\{(x, y) \in \mathbb{R}^2 | 0 < x^2 + y^2 < 4\}$ is an \emph{open} set.

Let $S$ denote the set and $o = (0, 0)$. For any point $p \in S$, let $\epsilon = \min(|p - o|, 2- |p - o|)$, which is clearly positive. Let $q = (x, y)$ be an arbitrary point in $B_{\epsilon}(p)$. Since
\begin{align*}
|q - o| &\leq |q - p| + |p - o| \\
        &<    \epsilon + |p - o| \\
        &\leq 2
\end{align*}
, $x^2 + y^2 < 4$. Since
\begin{align*}
|o - p| &\leq |o - q| + |q - p| \\
|o - q| &\geq |o - p| - |q - p| \\
        &\ge  |o - p| - \epsilon \\
        &\geq 0
\end{align*}
, $x^2 + y^2 > 0$. Therefore, $S$ is an open set. \QED

\item $\{(x, y) \in \mathbb{R}^2 | xy = 0\}$ is a \emph{closed} set.

Let $S$ denote the set and $S^C$ its complement $\mathbb{R}^2 - S$. Since $xy = 0$ iff $x = 0$ or $y = 0$, $S = \{(x,0) | x \in \mathbb{R}\} \cup \{(0, y) | y \in \mathbb{R}\}$. Let $p$ be an arbitrary point in $S^C$ and $\epsilon = \min(|x|, |y|)$. Clearly, $\epsilon > 0$. Let $q$ be an arbitrary point in $S$.

\begin{enumerate}
\item If $q = (x, 0)$, $|p - q| \geq \sqrt{(y - 0)^2} = |y| \geq \epsilon$, so $q \notin B_{\epsilon}(p)$.

\item If $q = (0, y)$, $|p - q| \geq \sqrt{(x - 0)^2} = |x| \geq \epsilon$, so $q \notin B_{\epsilon}(p)$.
\end{enumerate}

Therefore, for any point $q \in B_{\epsilon}(p)$, because $q \in \mathbb{R}^2$ and $q \notin S$, $q \in S^C$. Consequently, $S^C$ is an open set and $S$ is a closed set. \QED

\item $\{(x,y) \in \mathbb{R}^2 | 1 \leq x^2 + y^2 \leq 2\}$ is a \emph{closed} set.

Let $S$ denote the set, $S^C$ its complement $\mathbb{R}^2 - S$, and $o = (0, 0)$. Let $p = (x, y)$ be an arbitrary point of $S^C$.

\begin{enumerate}
\item If $x^2 + y^2 < 1$, let $\epsilon = 1 - |p - o|$, which is clearly positive. For any point $q = (x', y') \in B_{\epsilon}(p)$, $|q - o| \leq |q - p| + |p - o| \le \epsilon - |p - o| = 1$, so $x'^2 + y'^2 < 1$, or $q \notin S$.

\item If $x^2 + y^2 > 2$, let $\epsilon = |p - o| - \sqrt{2}$, which is also positive. For any point $q = (x', y') \in B_{\epsilon}(p)$, $|p - o| \leq |p - q| + |q - o|$. Then, $|q - o| \geq |p - o| - |p - q| > |p - o| - \epsilon = \sqrt{2}$. As a result, $x'^2 + y^2 > 2$, or $q \notin S$.
\end{enumerate}

This concludes that $S^C$ is open and consequently $S$ is closed. \QED

\end{enumerate}
\end{problem}

\begin{problem}{6}
\begin{enumerate}
\item Since $x^2 \leq x^2 + y^2$, and both sides are positive, we have:
\begin{align*}
0    &\leq \frac{x^2}{x^2 + y^2}  \leq 1   \\
-|x| &\leq \frac{x^2}{x^2 + y^2}x \leq |x| \\
\end{align*}
Note $\lim_{(x,y) \to (0, 0)} -|x| = \lim_{(x,y) \to (0, 0)} |x| = 0$. By squeeze theorem we obtain:
$$\lim_{(x,y) \to (0, 0)} \frac{x^3}{x^2 + y^2} = 0$$

\item Similar to (a), we have:
$$0 \leq \frac{x^2}{x^2+y^2}y^2 \leq y^2$$
Since $\lim_{(x,y) \to (0, 0)} 0 = \lim_{(x,y) \to (0, 0)} y^2 = 0$, by squeeze theorem we obtain:
$$\lim_{(x,y) \to (0, 0)} \frac{x^2y^2}{x^2 + y^2} = 0$$

\item Let $y = mx$, we get:
$$\frac{xy}{x^2 + y^2} = \frac{mx^2}{(1+m^2)x^2} = \frac{m}{1+m^2}$$
Since $m$ can vary, the limit doesn't exist.

\item Consider the limit with constraint $y = 2x$, we have:
$$\frac{x-y}{x^2+y^2} = -\frac{x}{2x^2} = -\frac{1}{2}x$$
whose limit doesn't exist as $x$ approaches $0$. Therefore, the limit of the original function doesn't exist.

\item Similar to (a), we have:
$$0 \leq \frac{y^2}{x^2+y^2}\sqrt{|x|} \leq \sqrt{|x|}$$
Since $\lim_{(x,y) \to (0, 0)} 0 = \lim_{(x,y) \to (0, 0)} \sqrt{|x|} = 0$, by squeeze theorem we obtain:
$$\lim_{(x,y) \to (0, 0)}\frac{y^2\sqrt{|x|}}{x^2+y^2} = 0$$

\item Consider the limit with constraint $x = y$, we have:
$$\frac{x\sqrt{|y|}}{x^2+y^2} = \frac{\sqrt{|y}}{2y}$$
whose limit doesn't exist as $y$ approaches $0$. Therefore, the limit of the original function doesn't exist.
\end{enumerate}
\end{problem}

\end{document}
