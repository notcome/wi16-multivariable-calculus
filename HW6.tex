\documentclass{homework}

\usepackage{mathpazo}

\title{Homework 6}

\begin{document}

\maketitle

\begin{problem}{1}
\begin{enumerate}
%%%% %%%% %%%% %%%% %%%% %%%% %%%% %%%% %%%% %%%% %%%% %%%% %%%% %%%% %%%% %%%%|
\item
$$[\mathbf{D}f\pmat{x\\y}] = \bmat{3x^2-12y & 24y^2-12x}$$
$$\left\{
\begin{aligned}
3x^2  &= 12y \\
24y^2 &= 12x
\end{aligned}
\right.$$
$$\left\{
\begin{aligned}
x^2  &= 4y \\
2y^2 &= x
\end{aligned}
\right.$$
$$4y^4 = y$$
$y$ could be $0$ or $1$. Therefore, the critical points are $(0, 0)$ and
$(2, 1)$.

%%%% %%%% %%%% %%%% %%%% %%%% %%%% %%%% %%%% %%%% %%%% %%%% %%%% %%%% %%%% %%%%|
\item
\begin{align*}
P_{f, (x,y)}^2 (\bvec{h})
&= f\pmat{x\\y} + \frac{1}{2}6xh_1^2 - 12h_1h_2
  + \frac{1}{2}24yh_2^2 - 12h_1h_2\\
&= f\pmat{x\\y} + 3xh_1^2 + 12yh_2^2 - 24h_1h_2\\
\end{align*}
When $(x, y) = (0, 0)$,
$$-24h_1h_2 = 6(h_1 - h_2)^2 - 24(\frac{1}{2}h_1 + \frac{1}{2}h_2)^2$$
, which has the signature $(1, 1)$. The point $(0, 0)$ is a saddle.

When $(x, y) = (2, 1)$,
$$6h_1^2 + 12h_2^2 - 24h_1h_2 = 6(h_1-2h_2)^2 - 12h_2^2$$
, which has the signature $(1, 1)$. The point $(2, 1)$ is a saddle.

\end{enumerate}
\end{problem}

\begin{problem}{2}
%%%% %%%% %%%% %%%% %%%% %%%% %%%% %%%% %%%% %%%% %%%% %%%% %%%% %%%% %%%% %%%%|
$$f(\mathbf{a}+\vec{\mathbf{h}}) = f(\mathbf{a}) +
Q_{f,\mathbf{a}}(\vec{\mathbf{h}}) + r(\vec{\mathbf{h}})
\quad \text{with} \quad \lim_{\vec{\mathbf{h}} \to \vec{\mathbf{0}}}
\frac{r(\vec{\mathbf{h}})}{|\vec{\mathbf{h}}|^2} = 0$$
Since $Q_{f,\mathbf{a}}(\vec{\mathbf{h}})$ has a signature $(k, l)$ where
$l > 0$, there exists a nontrivial subspace $W$ in which
$Q_{f,\mathbf{a}}(\vec{\mathbf{h}})$ is negative definite. Therefore, for
$\bvec{x} \in W$, $Q_{f,\mathbf{a}}(\bvec{x}) \leq -C|\bvec{x}|^2$ for some
$C > 0$, and
$$\frac{f(\mathbf{a} + (t\bvec{x}) - f(\mathbf{a})}{t^2}
= Q_{f,\mathbf{a}}(\bvec{x}) + \frac{r(t\bvec{x})}{t^2}
\leq -C|\bvec{x}|^2 + \frac{r(t\bvec{x})}{t^2}$$
. As a result, for $t$ sufficiently small, for
$\mathbf{p} = \mathbf{a} + t\bvec{x}$, $f(\mathbf{p}) < f(\mathbf{a})$, and
$\mathbf{p}$ exists in every neighborhood of $\mathbf{a}$ since $t$ can go
infinitely small. \QED
\end{problem}

\begin{problem}{3}
\begin{enumerate}
%%%% %%%% %%%% %%%% %%%% %%%% %%%% %%%% %%%% %%%% %%%% %%%% %%%% %%%% %%%% %%%%|
\item
$$[\mathbf{D}f\pmat{x\\y\\z}] = \bmat{y-z+yz, x+z+xz, y-x+xy}$$
$$\left\{
\begin{aligned}
y-z+yz &= 0\\
x+z+xz &= 0\\
y-x+xy &= 0
\end{aligned}
\right.$$
$x=y=z=0$ is a trivial solution. On the other hand, substitute $x$
and $y$ in the third equation by expressing them in terms of $z$:
$$\left\{
\begin{aligned}
x &= \frac{z}{1+z}\\
y &= -\frac{z}{1+z}
\end{aligned}
\right.$$
\begin{align*}
x+y+xy
&= 2\frac{z}{1+z}-(\frac{z}{1+z})^2 \\
&= \frac{z}{1+z}(2-\frac{z}{1+z})   \\
&= 0
\end{align*}
The other solution is $z = -2$. Therefore, there are two critical points,
$\pmat{0\\0\\0}$ and $\pmat{-2\\2\\-2}$.

%%%% %%%% %%%% %%%% %%%% %%%% %%%% %%%% %%%% %%%% %%%% %%%% %%%% %%%% %%%% %%%%|
\item
\begin{align*}
Q_{f, (x,y,z)}(\bpnt{h})
&= (1+z)h_1h_2 + (y-1)h_1h_3 \\
&+ (1+z)h_2h_1 + (1+x)h_2h_3 \\
&+ (1+x)h_3h_2 + (y-1)h_3h_1
\end{align*}
Let $x=0, y=0, z=0$,
\begin{align*}
Q_{f, (0,0,0)}(\bpnt{h})
&= 2h_1h_2 - 2h_1h_3 + 2h_2h_3 \\
&= 2(\frac{1}{2}h_1 + \frac{1}{2}h_2)^2
- \frac{1}{2}(x-y-2z)^2 + 2z^2
\end{align*}
. The quadratic form's signature is $(2, 1)$. Therefore the point is a saddle.

Let $x=-2,y=2,z=-2$,
\begin{align*}
Q_{f, (-2,2,-2)}(\bpnt{h})
&= -2h_1h_2 + 2h_1h_3 - 2h_2h_3
\end{align*}
, which is $-Q_{f, (0,0,0)}(\bpnt{h})$. Therefore, its signature is $(1,2)$,
and the point is also a saddle.

\end{enumerate}
\end{problem}

\begin{problem}{4}
\begin{enumerate}
%%%% %%%% %%%% %%%% %%%% %%%% %%%% %%%% %%%% %%%% %%%% %%%% %%%% %%%% %%%% %%%%|
\item
% $$F\pmat{x\\y\\z}=\pmat{x+y+z-2\\x+y-z-3}$$
% $$[\mathbf{D}F\pmat{x\\y\\z}] = \bmat{1&1&1\\1&1&-1}$$
% $$[\mathbf{D}f\pmat{x\\y\\z}] = \bmat{3x^2&3y^2&3z^2}$$
\begin{align*}
F\pmat{x\\y\\z}             &= \pmat{x+y+z-2\\x+y-z-3}\\
[\mathbf{D}F\pmat{x\\y\\z}] &= \bmat{1&1&1\\1&1&-1}\\
[\mathbf{D}f\pmat{x\\y\\z}] &= \bmat{3x^2&3y^2&3z^2}
\end{align*}
Using the method of Lagrange multipliers:
$$\left\{
\begin{aligned}
3x^2 = \lambda_1 + \lambda_2\\
3y^2 = \lambda_1 + \lambda_2\\
3z^2 = \lambda_1 - \lambda_2\\
x+y+z=2\\
x+y-z=3
\end{aligned}
\right.$$
First of all, $z = -\frac{1}{2}$. Since $x + y = \frac{5}{2} \neq 0$,
$x^2 = y^2$, we can conclude that $x = y = \frac{5}{4}$. Therefore, the
critical point is $\pmat{5/4\\5/4\\-1/2}$. Substituting these values yields that
$\lambda_1 = \frac{87}{32}$, $\lambda_2 = \frac{63}{32}$.

%%%% %%%% %%%% %%%% %%%% %%%% %%%% %%%% %%%% %%%% %%%% %%%% %%%% %%%% %%%% %%%%|
\item
Define the parametrization as
$$\gamma(x) = \pmat{x\\\frac{5}{2}-x\\2-x-(\frac{5}{2}-x)}
= \pmat{x\\\frac{5}{2}-x\\-\frac{1}{2}}$$
Then we have
\begin{align*}
f \circ \gamma(x)
&= x^3 + (\frac{5}{2} - x)^3 - \frac{1}{8} \\
&= x^3-x^3+\frac{15}{2}x^2-\frac{75}{4}x+\frac{125}{8}-\frac{1}{8} \\
&= \frac{15}{2}x^2-\frac{75}{4}x+\frac{31}{2} \\
&= \frac{15}{2}(x-\frac{5}{4})^2 + c
\end{align*}
for some $c$ that I am too lazy to compute. From high school algebra we know
that the critical point is a minimum.
\end{enumerate}
\end{problem}

\begin{problem}{5}
%%%% %%%% %%%% %%%% %%%% %%%% %%%% %%%% %%%% %%%% %%%% %%%% %%%% %%%% %%%% %%%%|
The unit sphere is $F^{-1}(0)$ where $F$ is defined as
$$F\pmat{x\\y\\z} = x^2+y^2+z^2-1$$
$$[\mathbf{D}F\pmat{x\\y\\z}] = \bmat{2x&2y&2z}$$
$$[\mathbf{D}f\pmat{x\\y\\z}] = \bmat{2y-4x&2x+2z-4y&2y-4z}$$
Then
$$\left\{
\begin{aligned}
y-2x=\lambda x\\
x+z-2y=\lambda y\\
y-2z=\lambda z\\
x^2+y^2+z^2=1
\end{aligned}
\right.$$
, which can be simplified as
$$\left\{
\begin{aligned}
y = (\lambda + 2)x\\
x+z = (\lambda + 2)y\\
y = (\lambda + 2)z\\
x^2+y^2+z^2=1
\end{aligned}
\right.$$

%%%% %%%% %%%% %%%% %%%% %%%% %%%% %%%% %%%% %%%% %%%% %%%% %%%% %%%% %%%% %%%%|
\begin{itemize}
\item If $x = 0$, then $y = 0$, $z = 0$. However, $(0, 0, 0)$ is not on the unit
sphere.
\item If $\lambda = - 2$, then $y = 0$, $x = -z$. From the constraint imposed
by the manifold we know $x = \sqrt{2}/2$, $z = -\sqrt{2}/2$.
\item Otherwise, rewrite $y$ in terms of $x$. Note $x = z$,
\begin{align*}
y &= (\lambda + 2)x \\
2x &= (\lambda + 2)y = (\lambda + 2)^2 x \\
1 &= 2x^2 + (\lambda + 2)^2x^2
\end{align*}
. $x = z = \pm 1/2$, $y = \pm \sqrt{2}/2$, and the signs do not interfere with
each other.
\end{itemize}
To sum up, there are five critical points,
$$
\pmat{{\sqrt{2}}/{2} \\ 0 \\ -{\sqrt{2}}/{2}},
\pmat{{1}/{2} \\ {\sqrt{2}}/{2} \\ {1}/{2}},
\pmat{{1}/{2} \\ -{\sqrt{2}}/{2} \\ {1}/{2}},
\pmat{-{1}/{2} \\ {\sqrt{2}}/{2} \\ -{1}/{2}},
\pmat{-{1}/{2} \\ -{\sqrt{2}}/{2} \\ -{1}/{2}}$$
\end{problem}

\begin{problem}{6}
%%%% %%%% %%%% %%%% %%%% %%%% %%%% %%%% %%%% %%%% %%%% %%%% %%%% %%%% %%%% %%%%|
The curve $C = F^{-1}(0)$ where $F$ is defined as
$$F\pmat{x\\y\\z} = \pmat{x^2+y^2-z^2\\x+y-z+1}$$
$$[\mathbf{D}F\pmat{x\\y\\z}] = \bmat{2x&2y&-2z\\1&1&-1}$$
We use the following function to measure the squared distance between a point
on $C$ and the origin:
$$f\pmat{x\\y\\z} = x^2+y^2+z^2$$
$$[\mathbf{D}f\pmat{x\\y\\z}] = \bmat{2x&2y&2z}$$
Then
$$\left\{
\begin{aligned}
2x &= 2x\lambda_1 + \lambda_2 \\
2y &= 2y\lambda_1 + \lambda_2 \\
-2z &= 2z\lambda_1 + \lambda_2 \\
z^2 &= x^2 + y^2 \\
z &= 1 + x + y
\end{aligned}
\right.$$
If $\lambda_1 = 1$ and $\lambda_2 = 0$, we have $z = 0$. Consequently,
$x = y = 0$ by $z^2 = x^2 + y^2$, but this does not satisfy $z = 1 + x + y$.

The other possibility is $x = y$. Using some substitution, we obtain
$$(2x+1)^2 - 2x^2 = 0$$
, which has two solutions,
$$x = -\frac{\sqrt{2} + 2}{2} \text{ or } x = \frac{\sqrt{2} - 2}{2}$$
. Therefore, there are two critical points:
$$\pmat{-(\sqrt{2} + 2)/2\\-(\sqrt{2} + 2)/2\\-(\sqrt{2} + 1)} \text{ and }
\pmat{(\sqrt{2}-2)/2\\(\sqrt{2}-2)/2\\\sqrt{2}-1}$$

One shall note that $C$ is a hyperbola in $\mathbb{R}^3$. Therefore, there is
no furthest point since $C$ goes to infinity. Geometrically, there should be
two local minima, which are consistent with the critical points we find. Some
simple calculation would yield that the second critical point above is the
closet one.
\end{problem}

\end{document}
