\documentclass{homework}

\usepackage{mathpazo}

\title{Homework 6}

\begin{document}

\maketitle

\begin{problem}{1}
\begin{enumerate}
%%%% %%%% %%%% %%%% %%%% %%%% %%%% %%%% %%%% %%%% %%%% %%%% %%%% %%%% %%%% %%%%|
\item
$$[\mathbf{D}f\pmat{x\\y}] = \bmat{3x^2-12y & 24y^2-12x}$$
$$\left\{
\begin{aligned}
3x^2  &= 12y \\
24y^2 &= 12x
\end{aligned}
\right.$$
$$\left\{
\begin{aligned}
x^2  = 4y \\
2y^2 = x
\end{aligned}
\right.$$
$$4y^4 = y$$
$y$ could be $0$ or $1$. Therefore, the critical points are $(0, 0)$ and
$(2, 1)$.

%%%% %%%% %%%% %%%% %%%% %%%% %%%% %%%% %%%% %%%% %%%% %%%% %%%% %%%% %%%% %%%%|
\item
\begin{align*}
P_{f, (x,y)}^2 (\mathbf{\vec{h}})
&= f\pmat{x\\y} + \frac{1}{2}6xh_1^2 - 12h_1h_2
  + \frac{1}{2}24yh_2^2 - 12h_1h_2\\
&= f\pmat{x\\y} + 3xh_1^2 + 12yh_2^2 - 24h_1h_2\\
\end{align*}
When $(x, y) = (0, 0)$,
$$-24h_1h_2 = 6(h_1 - h_2)^2 - 24(\frac{1}{2}h_1 + \frac{1}{2}h_2)^2$$
, which has the signature $(1, 1)$. The point $(0, 0)$ is a saddle.

When $(x, y) = (2, 1)$,
$$6h_1^2 + 12h_2^2 - 24h_1h_2 = 6(h_1-2h_2)^2 - 12h_2^2$$
, which has the signature $(1, 1)$. The point $(2, 1)$ is a saddle.

\end{enumerate}
\end{problem}

\begin{problem}{2}
%%%% %%%% %%%% %%%% %%%% %%%% %%%% %%%% %%%% %%%% %%%% %%%% %%%% %%%% %%%% %%%%|
$$f(\mathbf{a}+\vec{\mathbf{h}}) = f(\mathbf{a}) +
Q_{f,\mathbf{a}}(\vec{\mathbf{h}}) + r(\vec{\mathbf{h}})
\quad \text{with} \quad \lim_{\vec{\mathbf{h}} \to \vec{\mathbf{0}}}
\frac{r(\vec{\mathbf{h}})}{|\vec{\mathbf{h}}|^2} = 0$$
Since $Q_{f,\mathbf{a}}(\vec{\mathbf{h}})$ has a signature $(k, l)$ where
$l > 0$, there exists a nontrivial subspace $W$ in which
$Q_{f,\mathbf{a}}(\vec{\mathbf{h}})$ is negative definite. Therefore, for
$\bvec{x} \in W$, $Q_{f,\mathbf{a}}(\bvec{x}) \leq -C|\bvec{x}|^2$ for some
$C > 0$, and
$$\frac{f(\mathbf{a} + (t\bvec{x}) - f(\mathbf{a})}{t^2}
= Q_{f,\mathbf{a}}(\bvec{x}) + \frac{r(t\bvec{x})}{t^2}
\leq -C|\bvec{x}|^2 + \frac{r(t\bvec{x})}{t^2}$$
. As a result, for $t$ sufficiently small, for
$\mathbf{p} = \mathbf{a} + t\bvec{x}$, $f(\mathbf{p}) < f(\mathbf{a})$, and
$\mathbf{p}$ exists in every neighborhood of $\mathbf{a}$ since $t$ can go
infinitely small. \QED
\end{problem}

\begin{problem}{3}
\begin{enumerate}
%%%% %%%% %%%% %%%% %%%% %%%% %%%% %%%% %%%% %%%% %%%% %%%% %%%% %%%% %%%% %%%%|
\item
$$[\mathbf{D}f\pmat{x\\y\\z}] = \bmat{y-z+yz, x+z+xz, y-x+xy}$$
$$\left\{
\begin{aligned}
y-z+yz &= 0\\
x+z+xz &= 0\\
y-x+xy &= 0
\end{aligned}
\right.$$
$x=y=z=0$ is a trivial solution. On the other hand, substitute $x$
and $y$ in the third equation by expressing them in terms of $z$:
$$\left\{
\begin{aligned}
x &= \frac{z}{1+z}\\
y &= -\frac{z}{1+z}
\end{aligned}
\right.$$
\begin{align*}
x+y+xy
&= 2\frac{z}{1+z}-(\frac{z}{1+z})^2 \\
&= \frac{z}{1+z}(2-\frac{z}{1+z})   \\
&= 0
\end{align*}
The other solution is $z = -2$. Therefore, there are two critical points,
$\pmat{0\\0\\0}$ and $\pmat{-2\\2\\-2}$.

%%%% %%%% %%%% %%%% %%%% %%%% %%%% %%%% %%%% %%%% %%%% %%%% %%%% %%%% %%%% %%%%|
\item
\begin{align*}
Q_{f, (x,y,z)}(\bpnt{h})
&= (1+z)h_1h_2 + (y-1)h_1h_3 \\
&+ (1+z)h_2h_1 + (1+x)h_2h_3 \\
&+ (1+x)h_3h_2 + (y-1)h_3h_1
\end{align*}
Let $x=0, y=0, z=0$,
\begin{align*}
Q_{f, (0,0,0)}(\bpnt{h})
&= 2h_1h_2 - 2h_1h_3 + 2h_2h_3 \\
&= 2(\frac{1}{2}h_1 + \frac{1}{2}h_2)^2
- \frac{1}{2}(x-y-2z)^2 + 2z^2
\end{align*}
. The quadratic form's signature is $(2, 1)$. Therefore the point is a saddle.

Let $x=-2,y=2,z=-2$,
\begin{align*}
Q_{f, (-2,2,-2)}(\bpnt{h})
&= -2h_1h_2 + 2h_1h_3 - 2h_2h_3
\end{align*}
, which is $-Q_{f, (0,0,0)}(\bpnt{h})$. Therefore, its signature is $(1,2)$,
and the point is also a saddle.

\end{enumerate}
\end{problem}

\end{document}
