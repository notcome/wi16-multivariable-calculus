\documentclass{homework}

\title{Homework 5}

\begin{document}

\maketitle

\begin{problem}{1}
\begin{enumerate}
%%%% %%%% %%%% %%%% %%%% %%%% %%%% %%%% %%%% %%%% %%%% %%%% %%%% %%%% %%%% %%%%|
\item $S^2 - M = \{(x, 0, z) | x^2 + z^2 = 1, x \geq 0\}$

%%%% %%%% %%%% %%%% %%%% %%%% %%%% %%%% %%%% %%%% %%%% %%%% %%%% %%%% %%%% %%%%|
\item
$$[\mathbf{D}_1\gamma] = \bmat{-\sin u \cos v\\
                                \cos u \cos v\\
                                0}$$
$$[\mathbf{D}_2\gamma] = \bmat{-\cos u \sin v\\
                               -\sin u \sin v\\
                                \cos v}$$
$$[\mathbf{D}_1\gamma] \times [\mathbf{D}_2\gamma] =
\cos v \bmat{\cos u \cos v\\
             \sin u \cos v\\
             \sin v}$$

%%%% %%%% %%%% %%%% %%%% %%%% %%%% %%%% %%%% %%%% %%%% %%%% %%%% %%%% %%%% %%%%|
\item
$$\cos v \cdot \gamma\bmat{u\\v}
= [\mathbf{D}_1\gamma] \times [\mathbf{D}_2\gamma]$$
\end{enumerate}
\end{problem}

\begin{problem}{2}
\begin{enumerate}
%%%% %%%% %%%% %%%% %%%% %%%% %%%% %%%% %%%% %%%% %%%% %%%% %%%% %%%% %%%% %%%%|
\item $[\mathbf{DF}(x_1, x_2, x_3)] = \bmat{2x_1/r^2 & 2x_2/s^2 & 2x_3}$. Since
$x_1^2/r^2 + x_2^2/s^2 + x_3^2 = 1$ for all $(x_1, x_2, x_3) \in M$, at least
one entry is nonzero. Therefore, $[\mathbf{DF}(\mathbf{z})]$ is onto for all
$\mathbf{z} \in M$, and consequently $M$ is a $2$-dimensional manifold. \QED

%%%% %%%% %%%% %%%% %%%% %%%% %%%% %%%% %%%% %%%% %%%% %%%% %%%% %%%% %%%% %%%%|
\item $$\frac{2c_1}{r^2}\dot{x_1} +
        \frac{2c_2}{s^2}\dot{x_2} +
                    2c_3\dot{x_3} = 0$$

%%%% %%%% %%%% %%%% %%%% %%%% %%%% %%%% %%%% %%%% %%%% %%%% %%%% %%%% %%%% %%%%|
\item $$\vec{N_1} = \bmat{\frac{2c_1}{r^2}\\[4pt]
                          \frac{2c_2}{s^2}\\[4pt]
                          2c_3}$$
\end{enumerate}
\end{problem}

\begin{problem}{3}
\begin{enumerate}
%%%% %%%% %%%% %%%% %%%% %%%% %%%% %%%% %%%% %%%% %%%% %%%% %%%% %%%% %%%% %%%%|
\item
$$[\mathbf{D}_1\gamma] = \bmat{1\\0\\
-\frac{u}{r^2}\cdot\frac{1}{f(u, v)}}$$
$$[\mathbf{D}_2\gamma] = \bmat{0\\1\\
-\frac{v}{s^2}\cdot\frac{1}{f(u, v)}}$$
$$[\mathbf{D}_1\gamma] \times [\mathbf{D}_2\gamma] =
\bmat{
\frac{u}{r^2}\cdot\frac{1}{f(u, v)}\\
\frac{v}{s^2}\cdot\frac{1}{f(u, v)}\\
1}$$

%%%% %%%% %%%% %%%% %%%% %%%% %%%% %%%% %%%% %%%% %%%% %%%% %%%% %%%% %%%% %%%%|
\item
$$\frac{c_1}{r^2}\cdot\frac{1}{f(c_1, c_2)} \cdot \dot{x_1} +
  \frac{c_2}{s^2}\cdot\frac{1}{f(c_1, c_2)} \cdot \dot{x_2} +
  \dot{x_3} = 0$$

%%%% %%%% %%%% %%%% %%%% %%%% %%%% %%%% %%%% %%%% %%%% %%%% %%%% %%%% %%%% %%%%|
\item
$$\vec{N_2} = \bmat{
\frac{c_1}{r^2}\cdot\frac{1}{f(c_1, c_2)}\\
\frac{c_2}{s^2}\cdot\frac{1}{f(c_1, c_2)}\\
1}$$
Observe that in $\vec{N_1}$, $c_3 = f(c_1, c_2)$:
$$\vec{N_1} = 2\bmat{\frac{c_1}{r^2}\\\frac{c_2}{s^2}\\f(c_1, c_2)}$$
Clearly, $\vec{N_1} = 2f(c_1, c_2)\vec{N_2}$. \QED
\end{enumerate}
\end{problem}

\begin{problem}{4}
\begin{enumerate}
%%%% %%%% %%%% %%%% %%%% %%%% %%%% %%%% %%%% %%%% %%%% %%%% %%%% %%%% %%%% %%%%|
\item
$$[\mathbf{D}\gamma(\mathbf{x})] = \bmat{
1      & \hdots & 0      \\
\vdots & \ddots & \vdots \\
0      & \hdots & 1      \\
\mathbf{D}_1f_\mathbf{x} & \hdots & \mathbf{D}_{n-1}f_\mathbf{x}\\
}$$
Clearly,
$$\operatorname{span}(\mathbf{D}_1\gamma_{\mathbf{x}}, \dots,
\mathbf{D}_{n-1}\gamma_{\mathbf{x}}) = \operatorname{img}
[\mathbf{D}\gamma(\mathbf{x})] = T_{\gamma(\mathbf{x})}M$$
. Moreover, since
$\mathbf{D}_1\gamma_{\mathbf{x}}, \dots, \mathbf{D}_{n-1}\gamma_{\mathbf{x}}$
are linearly independent---note the first $n-1$ rows of 
$[\mathbf{D}\gamma(\mathbf{x})]$---they form a basis for 
$T_{\gamma(\mathbf{x})}M$. \QED

%%%% %%%% %%%% %%%% %%%% %%%% %%%% %%%% %%%% %%%% %%%% %%%% %%%% %%%% %%%% %%%%|
\item
For any vector $\mathbf{v} \in T_{\gamma(\mathbf{x})}M$, there exists $a_1,
\dots, a_{n-1}$ such that
$$\sum_{i=1}^{n-1}a_i\mathbf{D}_i\gamma(\mathbf{x}) = \mathbf{v}$$
Then, we have
$$\bmat{
-D_1f(\mathbf{x})\\
\vdots\\
-D_{n-1}f(\mathbf{x})\\
1} \cdot \bmat{
a_1\\
\vdots\\
a_{n-1}\\
\sum_{i=1}^{n-1}a_i\mathbf{D}_i\gamma(\mathbf{x})
} = -\sum_{i=1}^{n-1}a_i\mathbf{D}_i\gamma(\mathbf{x})
  +  \sum_{i=1}^{n-1}a_i\mathbf{D}_i\gamma(\mathbf{x})
  = 0$$
Therefore, this vector is perpendicular to $T_{\gamma(\mathbf{x})}M$. \QED

\end{enumerate}
\end{problem}

\begin{problem}{5}
\begin{enumerate}
%%%% %%%% %%%% %%%% %%%% %%%% %%%% %%%% %%%% %%%% %%%% %%%% %%%% %%%% %%%% %%%%|
\item
$$4+3x_2+4x_1x_3^2+2x_1x_2x_3^2+x_1^2x_2^2+2x_1^3x_3^2+3x_1^5$$

%%%% %%%% %%%% %%%% %%%% %%%% %%%% %%%% %%%% %%%% %%%% %%%% %%%% %%%% %%%% %%%%|
\item
$$\sum^3_{m=0}\sum_{I \in \mathcal{I}^m_3} a_I\mathbf{x}^I$$
where $a(0,1,0)=2,a(1,1,0)=1,a(1,1,1)=-1,a(2,0,0)=1,a(0,2,1)=5$, and all other
$a_I = 0$, for $I \in \mathcal{I}^m_3$ for $m \leq 3$.

%%%% %%%% %%%% %%%% %%%% %%%% %%%% %%%% %%%% %%%% %%%% %%%% %%%% %%%% %%%% %%%%|
\item
$$\sum^6_{m=0}\sum_{I \in \mathcal{I}^m_4} a_I\mathbf{x}^I$$
where $a(1,1,0,0)=3,a(0,1,1,1)=-1,a(0,2,1,0)=2,a(0,2,0,4)=1,a(0,5,0,0)=1$, and 
all other $a_I = 0$, for $I \in \mathcal{I}^m_3$ for $m \leq 6$.
\end{enumerate}
\end{problem}

\begin{problem}{6}
\begin{enumerate}
%%%% %%%% %%%% %%%% %%%% %%%% %%%% %%%% %%%% %%%% %%%% %%%% %%%% %%%% %%%% %%%%|
\item $x^2+4xy+4y^2 = (x+2y)^2$. Degenerate. Neither positive definitive nor
negative definitive, since for every $x \neq 0$, let $y = -1/2x$ and the
quadratic form would be 0.

%%%% %%%% %%%% %%%% %%%% %%%% %%%% %%%% %%%% %%%% %%%% %%%% %%%% %%%% %%%% %%%%|
\item $x^2+2xy+2y^2+2yz+z^2=(x+y)^2+(y+z)^2$. Degenerate. Neither positive
definitive nor negative definitive, since it is on $R^3$ but its signature
is $(2, 0)$.

%%%% %%%% %%%% %%%% %%%% %%%% %%%% %%%% %%%% %%%% %%%% %%%% %%%% %%%% %%%% %%%%|
\item
\begin{align*}
  & 2x^2+2y^2+z^2+w^2+4xy+2xz-2xw-2yw \\
= & x^2+z^2+2xz + x^2+2y^2+w^2+4xy-2w(x+y) \\
= & (x+z)^2 + w^2-2w(x+y)+(x+y)^2 - (x+y)^2 + x^2+2y^2+4xy \\
= & (x+z)^2+(w-x-y)^2 - x^2-y^2-2xy + x^2+2y^2+4xy \\
= & (x+z)^2+(w-x-y)^2 + y^2+2xy+x^2 - x^2 \\
= & (x+z)^2+(w-x-y)^2 + (y+x)^2 -x^2
\end{align*}
Clearly the four linear functions are linearly independent:
$$\bmat{1&-1&1&-1\\0&-1&1&0\\1&0&0&0\\0&1&0&0} \widetilde{}
\bmat{1&0&0&0\\0&1&0&0\\0&-1&1&0\\1&-1&1&-1} \widetilde{}
\bmat{1&0&0&0\\0&1&0&0\\0&0&1&0\\0&0&0&1}$$
It is nondegenerate, since it has a rank 4. It is neither positive definitive
nor negative definitive, since it is on $R^4$ but its signature is $(3,1)$.
\end{enumerate}
\end{problem}

\begin{problem}{7}
\begin{enumerate}
%%%% %%%% %%%% %%%% %%%% %%%% %%%% %%%% %%%% %%%% %%%% %%%% %%%% %%%% %%%% %%%%|
\item It's obvious that $S^{n-1} \subset B_2(\vec{0}) = \{\vec{x} \in 
\mathbb{R}^n : |\vec{x}| < 2\}$. Therefore, $S^{n-1}$ is bounded.
Let $\vec{p} \in \mathbb{R}^n \setminus S^{n-1}$. Let $\epsilon =
||\vec{p}| - 1|$. For any point $\vec{q} \in B_{\epsilon}(\vec{q})$,
$\vec{q} \notin S^{n-1}$, since
\begin{itemize}
    \item If $|\vec{p}| < 1$, $|\vec{q}| \leq |\vec{p}| + |\vec{p} - \vec{q}|
    \le 1$.
    \item If $|\vec{p}| > 1$, $|\vec{q}| \geq |\vec{p}| - |\vec{p} - \vec{q}|
    \ge 1$.
\end{itemize}
Therefore, $S^{n-1}$'s complement set is open and itself is closed.

This proves that $S^{n-1}$ is a compact set. By the existence of minima theorem,
there exists $\vec{x_0} \in S^{n-1}$ such that $Q(\vec{x_0}) \leq Q(\vec{x})$
for all $\vec{x} \in S^{n-1}$.

\item Since $\vec{x_0} \neq 0$ and $Q$ is a positive definitive quadratic form,
$Q(\vec{x_0}) > 0$.

\item \textbf{Claim.} The constant $C = Q(\vec{x_0})$.

Let $\vec{x} \in \mathbb{R}^n$. Since $|\frac{\vec{x}}{|\vec{x}|}| = 1$,
$\frac{\vec{x}}{|\vec{x}|} \in S^{n-1}$. Therefore,
$Q(\frac{\vec{x}}{|\vec{x}|}) \geq Q(\vec{x_0}) = C$. Consequently,
$Q(\vec{x}) = |\vec{x}|^2Q(\frac{\vec{x}}{|\vec{x}|}) \geq |\vec{x}|^2Q(C)$.
\QED
\end{enumerate}
\end{problem}

\end{document}